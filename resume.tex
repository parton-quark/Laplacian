\documentclass[a4j,10pt]{jsarticle}
\usepackage{layout,url,resume}
\usepackage[dvipdfmx]{graphicx}
\pagestyle{empty}

\begin{document}
%\layout

\title{ Laplacian in Cylindrical coordinate system}

% 和文著者名
\author{
    parton \thanks{aqua}
}

% 和文概要
%\begin{abstract}
%\end{abstract}

\maketitle
\thispagestyle{empty}

\section{Laplacian}
Cartesian coordinates are given as $(X, Y, Z)$.

In this system,  laplacian is defined as
$$\Delta f = \frac{\partial ^ 2 f}{\partial X^2} + \frac{\partial ^ 2 f}{\partial Y^2} + \frac{\partial ^ 2 f}{\partial Z^2}$$

\section{Cylindrical coordinate system}
Cylindrical coordinates are given as $(r, \theta, z)$.
\subsection{mapping}
Cartesian coordinates and cylindrical coordinates are mapped as follows
\begin{eqnarray}
X &=& r\cos \theta \\
Y &=& r\sin \theta \\
Z &=&z 
\end{eqnarray}
By using this definition, 
\begin{eqnarray}
r  &=& {(X^2 + Y^2)}^\frac{1}{2} \\
\theta &=& \tan^-1 \frac{Y}{X} \\
z  &=& Z 
\end{eqnarray}


\subsection{Preparation}
By (4), 
\begin{equation}
\frac{\partial r}{\partial X} = \frac{\partial}{\partial X} {(X^2 + Y^2)}^\frac{1}{2}\\ 
=  X{(X^2 + Y^2)}^-\frac{1}{2}\\
=\cos \theta \\
\end{equation}
\begin{equation}
\frac{\partial r}{\partial X} = \frac{\partial}{\partial Y} {(X^2 + Y^2)}^\frac{1}{2}\\ 
=  Y{(X^2 + Y^2)}^-\frac{1}{2}\\
=\sin \theta \\
\end{equation}
\begin{equation}
\frac{\partial r}{\partial X} = \frac{\partial}{\partial Z} {(X^2 + Y^2)}^\frac{1}{2} = 0 
\end{equation}

By (5)\footnote{$$\frac{\partial }{\partial u} \tan ^ {-1} u = \frac{1}{1+u^2} $$}, 
\begin{equation}
\frac{\partial \theta}{\partial X} = \frac{\partial}{\partial X} \tan ^ {-1} \frac{Y}{X} \\
= -\frac{Y}{X^2 + Y^2}\\
= - \frac{1}{r}\sin\theta
\end{equation}

\begin{equation}
\frac{\partial \theta}{\partial Y} = \frac{\partial}{\partial Y} \tan ^ {-1} \frac{Y}{X} \\
= \frac{Y}{X^2 + Y^2}\\
= \frac{1}{r}\cos \theta
\end{equation}

\begin{equation}
\frac{\partial \theta}{\partial Z} = \frac{\partial}{\partial Z} \tan ^ {-1} \frac{Y}{X} \\
= 0
\end{equation}

By (6), 
\begin{equation}
\frac{\partial z}{\partial X}  = 0
\end{equation}
\begin{equation}
\frac{\partial z}{\partial Y}  = 0
\end{equation}
\begin{equation}
\frac{\partial z}{\partial Z}  = 1
\end{equation}

\section{function in cylindrical coordinate system}
Let $f$ be a function in cylindrical coordinate system.
$$f = (r, \theta,  z)$$
Cylindrical coordinate system can be regarded as a function of orthogonal coordinates\footnote{(4),(5), and (6)}. Therefore, function $f$ can be regarded as a composite function.
\begin{equation}
f(r(X, Y, Z), \theta(X, Y, Z), z(X, Y, Z))
\end{equation}
Partially differentiate equation (16) with X, Y, Z. 

\begin{equation}
\frac{\partial f}{\partial X}=\frac{\partial r}{\partial X} \frac{\partial f}{\partial r}+\frac{\partial \theta}{\partial X} \frac{\partial f}{\partial \theta}+\frac{\partial z}{\partial X} \frac{\partial f}{\partial z}
\end{equation}
\begin{equation}
\frac{\partial f}{\partial Y}=\frac{\partial r}{\partial Y} \frac{\partial f}{\partial r}+\frac{\partial \theta}{\partial Y} \frac{\partial f}{\partial \theta}+\frac{\partial z}{\partial Y} \frac{\partial f}{\partial z}
\end{equation}
\begin{equation}
\frac{\partial f}{\partial Z}=\frac{\partial r}{\partial Z} \frac{\partial f}{\partial r}+\frac{\partial \theta}{\partial Z} \frac{\partial f}{\partial \theta}+\frac{\partial z}{\partial Z} \frac{\partial f}{\partial z}
\end{equation}
Then,
\begin{eqnarray*}
&\frac{\partial^2 f}{\partial X^2} =\frac{\partial}{\partial X}\frac{\partial f}{\partial X} \\
&=\frac{\partial r}{\partial X}\frac{\partial}{\partial r}\frac{\partial f}{\partial X} + \frac{\partial \theta}{\partial X}\frac{\partial}{\partial \theta}\frac{\partial f}{\partial X} + \frac{\partial z}{\partial X}\frac{\partial}{\partial z}\frac{\partial f}{\partial X} \\
&= \frac{\partial r}{\partial X}\frac{\partial}{\partial r}(17) +\frac{\partial \theta}{\partial X}\frac{\partial}{\partial \theta}(17) + \frac{\partial z}{\partial X}\frac{\partial}{\partial z}(17)
\end{eqnarray*}

\begin{eqnarray*}
&\frac{\partial^2 f}{\partial Y^2} =\frac{\partial}{\partial Y}\frac{\partial f}{\partial Y} \\
&=\frac{\partial r}{\partial Y}\frac{\partial}{\partial r}\frac{\partial f}{\partial Y} + \frac{\partial \theta}{\partial Y}\frac{\partial}{\partial \theta}\frac{\partial f}{\partial Y} + \frac{\partial z}{\partial Y}\frac{\partial}{\partial z}\frac{\partial f}{\partial Y} \\
&= \frac{\partial r}{\partial Y}\frac{\partial}{\partial r}(18) +\frac{\partial \theta}{\partial Y}\frac{\partial}{\partial \theta}(18) + \frac{\partial z}{\partial Y}\frac{\partial}{\partial z}(18)
\end{eqnarray*}

\begin{eqnarray*}
&\frac{\partial^2 f}{\partial Z^2} =\frac{\partial}{\partial Z}\frac{\partial f}{\partial Z} \\
&=\frac{\partial r}{\partial Z}\frac{\partial}{\partial r}\frac{\partial f}{\partial Z} + \frac{\partial \theta}{\partial Z}\frac{\partial}{\partial \theta}\frac{\partial f}{\partial Z} + \frac{\partial z}{\partial Z}\frac{\partial}{\partial z}\frac{\partial f}{\partial Z} \\
&= \frac{\partial r}{\partial Z}\frac{\partial}{\partial r}(19) +\frac{\partial \theta}{\partial Z}\frac{\partial}{\partial \theta}(19) + \frac{\partial z}{\partial Z}\frac{\partial}{\partial z}(19)
\end{eqnarray*}

Using these equations, laplacian  is
\begin{eqnarray*}
&\Delta f = \frac{\partial^2 f}{\partial Z^2} \\ 
&= \frac{\partial r}{\partial X}\frac{\partial}{\partial r}(17) +\frac{\partial \theta}{\partial X}\frac{\partial}{\partial \theta}(17) + \frac{\partial z}{\partial X}\frac{\partial}{\partial z}(17) \\
& + \frac{\partial r}{\partial Y}\frac{\partial}{\partial r}(18) +\frac{\partial \theta}{\partial Y}\frac{\partial}{\partial \theta}(18) + \frac{\partial z}{\partial Y}\frac{\partial}{\partial z}(18) \\
&  + \frac{\partial r}{\partial Z}\frac{\partial}{\partial r}(19) +\frac{\partial \theta}{\partial Z}\frac{\partial}{\partial \theta}(19) + \frac{\partial z}{\partial Z}\frac{\partial}{\partial z}(19)
\end{eqnarray*}

Substitute (7)〜(15) to this equation and got 
\begin{eqnarray*}
&\Delta f = \cos \theta \frac{\partial}{\partial r} (\cos \theta \frac{\partial f}{\partial r} - \frac{1}{r}\sin \frac{\partial f}{\partial \theta})\\
&-\frac{1}{r}\sin\theta \frac{\partial}{\partial \theta} (\cos\theta \frac{\partial f}{\partial r} - \frac{1}{r}\sin\theta \frac{\partial f}{\partial \theta})\\
&+\sin \theta \frac{\partial}{\partial r} (\sin \theta \frac{\partial f}{\partial r} + \frac{1}{r}\cos \theta \frac{\partial f}{\partial \theta} )\\
&+\frac{1}{r}\cos \theta \frac{\partial}{\partial \theta} (\sin \theta \frac{\partial f}{\partial r} + \frac{1}{r}\cos\theta \frac{\partial f}{\partial \theta}) \\
&+ \frac{\partial}{\partial z} (\frac{\partial f}{\partial z})
\end{eqnarray*}
Transform the equation 
\begin{equation}
\Delta f = \frac{1}{r}\frac{\partial }{\partial r}(r \frac{\partial f}{\partial r})+\frac{1}{r^2}\frac{\partial ^2 f}{\partial \theta^2} + \frac{\partial ^2 f}{\partial z^2}
\end{equation}

\section{参考}
埼玉工業大学 機械工学学習セミナー(小西克享) テーマB27 ラプラシアンの直交座標から円柱座標への座標変換(2016)

\bibliographystyle{junsrt}
\bibliography{resume}

\end{document}
% end of file
